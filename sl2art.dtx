% \iffalse meta-comment
%
% Copyright (C) 2024 Calvin McPhail-Snyder
% --------------------------------
%
% This file may be distributed and/or modified under the
% conditions of the LaTeX Project Public License, either version 1.3
% of this license or (at your option) any later version.
% The latest version of this license is in:
%
%
% http://www.latex-project.org/lppl.txt
%
% and version 1.3c or later is part of all distributions of LaTeX
% version 2008-05-04 or later.
%
% \fi
%
% \iffalse
%<*driver>
\ProvidesFile{sl2art.dtx}
%</driver>
%<class>\NeedsTeXFormat{LaTeX2e}[2023-11-01]
%<class>\ProvidesExplClass{sl2art}{2024-04-28}{v0.3}{a one-and-a-half-column article class for mathematics}
%
%<*driver>
\documentclass{ltxdoc}
\usepackage{xcolor,amssymb}
\EnableCrossrefs
\CodelineIndex
\RecordChanges
\begin{document}
  \DocInput{sl2art.dtx}
\end{document}
%</driver>
% \fi
%
% \changes{v0.1}{2024-04-22}{Initial version}
%
% \GetFileInfo{sl2art.dtx}
%
% \parskip=12pt
%
% \title{The \textsf{sl2art} class}
% \author{Calvin McPhail-Snyder \\ \texttt{calvin@sl2.site}}
%
% \maketitle
% \begin{abstract}
%   This is a document class for writing mathematics notes and papers.
%   It uses a ``one and a half column'' layout: the text block is shifted to one side to allow use of the wide margin for notes, citations, and small figures.
% \end{abstract}
%
% \section{Introduction}
%
% This is a document class for mathematics papers.
% It is probably easiest to see how to use this class by looking at some of the example files; |example.tex| talks more about design choices.
%   This class is not particularly customizable, since the goal was to make a class for how I like to do things.
%   However, you are welcome to use it as a starting point for creating your own styles.
%
% \PrintChanges
% 
% \section{Usage}
%
% \DescribeMacro{\maketitle}
% By default we use a title page.
% To add author information use |\author| and optionally |\address| (for affiliations) and |\email|, with one invocation for each author.
% The abstract is set in the |abstract| environment, which can be placed anywhere before calling |\maketitle|, but ideally should be set after |\begin{document}| for better compatibility with the |sl2ams| compatibility layer.
%
% \DescribeMacro{\defemph}
% For definitional emphasis; in addition to italics, we use the accent color to make key words stand out.
%
% \DescribeMacro{\note}
% Create a sidenote, which we use instead of footnotes.
%
% \DescribeMacro{\marginfigure}
% Taken from the |sidenotes| package, for placing figures in the margin.
% These should be at most 2in wide to fit.
% One can also use |\margintable| for tables.
%
% \DescribeMacro{\declaretheorem}
% We use |thmtools| for creating theorem-like environments.
% By default there are three styles.
% |theorem| and |definition| are almost identical, but |theorem| uses a \(\lrcorner\) to indicate the end of the theorem (since we \emph{don't} use italicized text for theorem statements).
% The intent is |theorem| for theorems, lemmas, etc. and |definition| for definitions, remarks, and examples.
% Both are numbered by section.
% |bigtheorem| is identical to |theorem| but by default has its own global numbering; the idea is that major results are just "Theorem 1, Theorem 2" and so on.
%
% \definecolor{accent}{HTML}{914E0F}
% \definecolor{link}{HTML}{105C9C}
% \definecolor{slred}{HTML}{FA3F2D}
% \definecolor{slyellow}{HTML}{F08813}
% \definecolor{slgreen}{HTML}{01F556}
% \definecolor{slblue}{HTML}{30A4FF}
% \definecolor{slpurple}{HTML}{AD2BF0}
% \DescribeMacro{colorpalette}
% In addition to some colors {\color{accent} \texttt{accent}} (for section headers, emphasis, etc.) and {\color{link} \texttt{link}} (for hyperlinks) we define
% \begin{itemize}
%   \item {\color{slred} \texttt{slred FA3F2D}} 
%   \item {\color{slyellow} \texttt{slyellow F08813}} 
%   \item {\color{slgreen} \texttt{slgreen 01F556}} 
%   \item {\color{slblue} \texttt{slblue 30A4FF}} 
%   \item {\color{slpurple} \texttt{slpurple AD2BF0}} 
% \end{itemize}
% for use in figures, etc.
% According to color.adobe.com this is a colorblind-safe palette.
%
% \section{Package options}
% \DescribeMacro{draft}
% Adds a message to the title block that the document is a draft.
% |false| by default.
%
% \DescribeMacro{margincite}
% Whether to print short bibliographic info in the margin on first citation.
% |true| by default.
%
% \DescribeMacro{titlepage}
% Whether to have a separate title page.
% |true| by default.
%
% \section{Implementation}
%
% Set up keys for package options.
%    \begin{macrocode}
\RequirePackage{expl3,l3keys2e}
\keys_define:nn{ sl }
  {
    draft .bool_set:N = \g_sl_draft_bool,
    draft .initial:n = false,
    draft .default:n = true,
    margincite .bool_set:N = \g_sl_margincite_bool,
    margincite .initial:n = true,
    margincite .default:n = true,
    titlepage .bool_set:N = \g_sl_titlepage_bool,
    titlepage .initial:n = true,
    titlepage .default:n = true,
  }
\ProcessKeysOptions{ sl }
%    \end{macrocode}
%
% We use \texttt{article} as a base class.
%    \begin{macrocode}
\LoadClass[10pt,oneside]{article}
%    \end{macrocode}
%
% Set page dimensions using the geometry package.
%    \begin{macrocode}
\RequirePackage[
  includeall,
  marginparwidth=2in,
  marginparsep=0.25in,
  width=7.5in,
  height=9in
]{geometry}
%    \end{macrocode}
%
% Libertine fonts for body text, Computer Modern for math, and Bera Mono for monospaced text.
% Bera Mono is scaled to 80\% to better match the other fonts.
%    \begin{macrocode}
\RequirePackage[T1]{fontenc}
\RequirePackage{microtype}
\RequirePackage{libertine}
\RequirePackage[scaled=0.80]{beramono}
%    \end{macrocode}
%
% \begin{macro}{colorpalette}
% A standard set of colors for use in the document.
%    \begin{macrocode}
\RequirePackage{xcolor}
\definecolor{accent}{HTML}{914E0F}
\definecolor{link}{HTML}{105C9C}
\definecolor{slred}{HTML}{FA3F2D}
\definecolor{slyellow}{HTML}{F08813}
\definecolor{slgreen}{HTML}{01F556}
\definecolor{slblue}{HTML}{30A4FF}
\definecolor{slpurple}{HTML}{AD2BF0}
%    \end{macrocode}
% \end{macro}
% 
% \begin{macro}{\defemph}
% We use the standard |\emph| command to handle italicized environments, although we don't use those.
%    \begin{macrocode}
\NewDocumentCommand{\defemph}{m}{{\color{accent}\emph{#1}}}
%    \end{macrocode}
% \end{macro}
%
% \begin{macro}{\note}
% Use |sidenotes| for margin notes.
% We define |\@captype| so that other macros can detect if we are inside a float.
% This is used by other macros to tell if we are inside a figure, table, etc., which is why it's not a \texttt{latex3} variable.
%    \begin{macrocode}
\RequirePackage[oneside]{sidenotes}
\RequirePackage{marginfix}
\NewDocumentCommand{\note}{m}{
  \sidenote{
    \def\@captype{sidenote}
    \footnotesize #1
    \undef\@captype
  }
}
%    \end{macrocode}
%
% \end{macro}
% \begin{macro}{sectioning}
% Next we set up headers, footers, and sections, along with their associated TOC entries.
% \changes{v0.3}{2024-03-20}{Bugfix: stop title color from running into margin in certain cases}
%    \begin{macrocode}
\RequirePackage{titletoc}
\titlecontents{part}[1.5em]
  {\vspace{2pt}}
  {\scshape \contentslabel{2.3em}}
  {\scshape Part }
  {\titlerule*[1pc]{}\contentspage}[\vspace{2pt}]
\titlecontents{section}[3.8em] % ie, 1.5em (part) + 2.3em
  {}
  {\contentslabel{2.3em}}
  {\hspace*{-2.3em}}
  {\titlerule*[1pc]{}\contentspage}
\titlecontents{subsection}[5.3em] % ie, 3.8em (section) + 2.3em
  {}
  {\contentslabel{2.3em}}
  {\hspace*{-2.3em}}
  {\titlerule*[1pc]{}\contentspage}
\titlecontents{subsubsection}[7.6em] % ie, 5.3em (subsection) + 2.3em
  {}
  {\contentslabel{2.3em}}
  {\hspace*{-2.3em}}
  {\titlerule*[1pc]{}\contentspage}

\RequirePackage[pagestyles]{titlesec}
\titleformat{\part}{\scshape}{Part \thepart:\@}{.5em}{}
\titleformat{\section}{\scshape }{\color{accent} \thesection.\@}{.5em}{\color{accent}}
\titleformat{\subsection}{\itshape }{\color{accent} \thesubsection.\@}{.5em}{\color{accent}}
\titleformat{\subsubsection}[runin]{}{\color{accent} \thesubsubsection.\@}{.5em}{\color{accent}}

\newpagestyle{main}{
  \sethead[][][]{\scshape \ifthesection{\thesection.}{}~\sectiontitle}{}{\thepage}
}
\pagestyle{main}
%    \end{macrocode}
% \end{macro}
%
% Use small caps for figure, etc. labels
%    \begin{macrocode}
\renewcommand{\descriptionlabel}[1]{\hspace{\labelsep}\scshape{#1}}
%    \end{macrocode}
%
%
% \begin{macro}{\cite}
% Set up citations. By default we use a custom version of |alphabetic| that puts mini-citations in the sidebar.
% There is a package option to turn this off.
%    \begin{macrocode}
\bool_if:NTF \g_sl_margincite_bool
{
  \RequirePackage[style=alphabetic-side,maxbibnames=5]{biblatex}
}
{
  \RequirePackage[style=alphabetic,maxbibnames=5]{biblatex}
}
%    \end{macrocode}
% \end{macro}
%
% \begin{macro}{\declaretheorem}
% Theorem styles for theorems/lemmas/etc., important theorems (which get their own counters) and defintions/remarks/etc.
% We need to load |amssymb| to use |\lrcorner|.
%    \begin{macrocode}
\RequirePackage{amsthm,amssymb}
\RequirePackage{thmtools}
\declaretheoremstyle[
  headfont=\scshape,
  qed=$\lrcorner$,
  parent=section,
]{theorem}
\declaretheoremstyle[
  headfont=\scshape,
  qed=$\lrcorner$,
  name='Theorem'
]{bigtheorem}
\declaretheoremstyle[
  headfont=\scshape,
  parent=section,
]{definition}
%    \end{macrocode}
% \end{macro}
%
% \begin{macro}{\maketitle}
% Next we handle the title page and abstract.
% First we have code for defining the title, subject class, keywords, and abstract (which is an enviornment to match |amsart|)
% \changes{v0.2}{2024-04-28}{Add subject class and keyword support}
%    \begin{macrocode}
\tl_new:N \g_sl_title_tl
\RenewDocumentCommand{\title}{m}{ \tl_set:Nn \g_sl_title_tl {#1} }

\tl_new:N \g_sl_subjclass_tl
\tl_new:N \g_sl_subjclassYear_tl
\NewDocumentCommand{\subjclass}{O{2020} m}{ 
  \tl_set:Nn \g_sl_subjclass_tl {#2} 
  \tl_set:Nn \g_sl_subjclassYear_tl {#1} 
}

\tl_new:N \g_sl_keywords_tl
\NewDocumentCommand{\keywords}{m}{ \tl_set:Nn \g_sl_keywords_tl {#1} }

\tl_new:N \g_sl_abstract_tl
\RenewDocumentEnvironment{ abstract }{ +b }{ \tl_gset:Nn \g_sl_abstract_tl {#1} }{ }
%    \end{macrocode}
% Next we deal with the authors.
% To match |amsart| each invocation adds another author.
% To better handle omitted |\address| or |\email| commands every invocation of |\author| extends all three arrays, and the other two commands just overwrite the last (blank) entry.
%    \begin{macrocode}
\seq_new:N \g_sl_author_seq
\seq_new:N \g_sl_address_seq
\seq_new:N \g_sl_email_seq
\RenewDocumentCommand{\author}{m}{
  \seq_put_right:Nn \g_sl_author_seq {#1}
  \seq_put_right:Nn \g_sl_address_seq {}
  \seq_put_right:Nn \g_sl_email_seq {}
}
\NewDocumentCommand{\address}{m}{
  \seq_set_item:Nnn \g_sl_address_seq {-1} {#1}
}
\NewDocumentCommand{\email}{m}{
  \seq_set_item:Nnn \g_sl_email_seq {-1} {#1}
}
%    \end{macrocode}
%
% Finally, code for producing the title page.
% We have some local variables for storing the current author data during our loop.
%    \begin{macrocode}
\tl_new:N \l_sl_curr_author_tl
\tl_new:N \l_sl_curr_address_tl
\tl_new:N \l_sl_curr_email_tl
\cs_gset:Nn \g_sl_maketitle:
%    \end{macrocode}
% This is slightly ``clever'' (pejorative) but it seems bad to have two copies of the (somewhat complicated) code defining the title block.
% Instead we use conditions to turn on and off |titlepage| and the page geometry when a separate title page was requested (the default behavior).
%    \begin{macrocode}
{
  \bool_if:NTF \g_sl_titlepage_bool
  {
    \begin{titlepage}
    \newgeometry{margin=1.5in}
  }{}
%    \end{macrocode}
% Title
%    \begin{macrocode}
    \thispagestyle{empty}
    \parindent=0pt
    \parskip=12pt
    { \large \scshape \g_sl_title_tl }
    \par

%    \end{macrocode}
% Deal with the authors
%    \begin{macrocode}
    \begin{minipage}{\textwidth}
      \bool_until_do:nn {\seq_if_empty_p:N \g_sl_author_seq} {
        \seq_pop_left:NN \g_sl_author_seq {\l_sl_curr_author_tl}
        \seq_pop_left:NN \g_sl_address_seq {\l_sl_curr_address_tl}
        \seq_pop_left:NN \g_sl_email_seq {\l_sl_curr_email_tl}
        \l_sl_curr_author_tl
        \tl_if_empty:NTF \l_sl_curr_address_tl {} {
          \par
          {\scshape \l_sl_curr_address_tl}
        }
        \tl_if_empty:NTF \l_sl_curr_email_tl {} {
          \par
          \href{mailto:\l_sl_curr_email_tl}{\l_sl_curr_email_tl}
        }
        \\[10pt]
      }
      \vspace{-2em}
    \end{minipage}

%    \end{macrocode}
% Abstract. If the variable is empty (because it was not set by the |abstract| environment) don't print anything.
%    \begin{macrocode}

    \tl_if_empty:NTF \g_sl_abstract_tl {} {
      \begin{center}
      \begin{minipage}{4in} 
      {\centering \scshape \color{accent} Abstract \\ }
        \g_sl_abstract_tl
      \end{minipage}
      \end{center}
    }
%    \end{macrocode}
% When |draft| is enabled, add a message to the title block.
%    \begin{macrocode}
    \bool_if:NTF \g_sl_draft_bool {
      \begin{center}
        {\color{slred} \large DRAFT:~Not~for~distribution }
      \end{center}
    } {}
%    \end{macrocode}
% Add MSC and keywords. If there's a title page, these go at the bottom (so there's a |\vfill|); if not we have a small break instead.
%    \begin{macrocode}
  \bool_if:NTF \g_sl_titlepage_bool {\vfill}{\par}
  \tl_if_empty:NTF \g_sl_subjclass_tl {} {
    {\itshape {\g_sl_subjclassYear_tl}~Mathematics~Subject~Classification:~}
    \g_sl_subjclass_tl
    \par
  }
  \tl_if_empty:NTF \g_sl_keywords_tl {} {{\itshape Key~words~and~phrases:}~ \g_sl_keywords_tl }
%    \end{macrocode}
% Change pagaraph spacing back to default (if we don't do this manually it causes problems when there isn't a title page).
% If we have a separate titlepage we also need to close the |titlepage| environment.
%    \begin{macrocode}
  \bool_if:NTF \g_sl_titlepage_bool
  {
    \end{titlepage}
  }{}
  \parskip=0pt
  \parindent=1.5em
}
\RenewDocumentCommand{\maketitle}{}{\g_sl_maketitle:}
%    \end{macrocode}
% \end{macro}
%
% We conclude with |hyperref| setup.
%    \begin{macrocode}
\RequirePackage{hyperref}
\hypersetup{
  colorlinks=true,
  citecolor=link,
  linkcolor=link,
  menucolor=link,
  linktocpage=true,
  urlcolor=link,
  draft=false
}
%    \end{macrocode}
%
% \Finale
\endinput
